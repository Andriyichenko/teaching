textbf{問2}\qquad 次の文中の $\ab{\textrm{F}}$、$\ab{\textrm{J}}$ には、適する数を入れ、その他の \ab{\ } には右のぺージの選択肢 \ding{172} 〜 \ding{181} の中から適するものを選びなさい。

\vspace{1em}

A、B、C の3つの箱に、それぞれ9枚のカードが入っている。9枚のカードには、1から9の数字が1つずつ書かれている。

\vspace{1em}

A、B、C の箱から順に1枚ずつカードを取り出し、それらのカードに書かれた数字をそれぞれ $a$、$b$、$c$ とする。

\vspace{0.5em}

(i)\qquad $a = b \neq c$ である確率は \ab{\textrm{H}} である。

\vspace{0.5em}

(ii)\qquad $a$、$b$、$c$ の中に同じものがない確率は \ab{\textrm{I}} である。

\vspace{0.5em}

(iii)\qquad $a > 2b > 3c$ となる確率$p$を求めよう。

\quad\quad\ \ \ \  条件を満たす $a$、$b$、$c$ の組が存在するようなbの範囲は

\[\ab{\textrm{J}} \leq b \leq \ab{\textrm{K}}\]

である。このことに注目して

\[p = \ab{\textrm{L}}\]

を得る。

\vspace{1em}

A から 3枚、B から 2枚、合わせて 5枚のカードを取り出す。

\vspace{0.5em}

(i) 取り出した 5枚のカードの中に奇数が4枚、偶数が1枚ある確率は \ab{\textrm{M}} である。

\vspace{0.5em}

(ii) 取り出した 5枚のカードの中に奇数が少なくとも1枚ある確率は \ab{\textrm{N}} である。

% \vspace{1em}

% \ \ \ \ \ \ \ \ \ \ \ \qquad \qquad\qquad\qquad\qquad\qquad\qquad\qquad\qquad\quad\quad\textbf{(問2は次ページに続く)}

\vspace{2em}


% \begin{tabular}{|c|c|c|c|c|c|c|c|}
% \hline
% \ding{173} & $\frac{8}{81}$ & ① & $\frac{14}{81}$ & ② & $\frac{56}{81}$ & ③ & $\frac{64}{81}$ \\
% \hline
% ④ & $\frac{25}{126}$ & ⑤ & $\frac{35}{126}$ & ⑥ & $\frac{115}{126}$ & ⑦ & $\frac{125}{126}$ \\
% \hline
% ⑧ & $\frac{8}{729}$ & ⑨ & $\frac{10}{729}$ & & & & \\
% \hline
% \end{tabular}

\noindent  
\ding{172}\ $ \dfrac{8}{81} $\quad\quad\  \ \ding{173}\ $ \dfrac{14}{81} $\qquad
\ \ \ding{174}\ $ \dfrac{56}{81} $\qquad \ \ding{175}\ $ \dfrac{64}{81} $
\\
\\
\ding{176}\ $ \dfrac{25}{126} $\qquad \ding{177}\ $ \dfrac{35}{126} $\qquad
\ding{178}\ $ \dfrac{115}{126} $\qquad \ding{179}\ $ \dfrac{125}{126} $ \\
\\
\ding{180}\ $ \dfrac{8}{729} $\qquad \ding{181}\ $ \dfrac{10}{729} $
