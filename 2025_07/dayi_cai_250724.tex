\documentclass[dvipdfmx,a4paper,12pt]{jarticle}
\usepackage{amsmath,amssymb,amsfonts,amsthm}
\usepackage[dvipdfmx]{graphicx}
\usepackage{tikz}
\usetikzlibrary{positioning, intersections, calc, arrows.meta, math, through}
\usepackage{tcolorbox}
\tcbuselibrary{theorems,breakable}
\usepackage{enumerate}
\usepackage{mathtools}
\usepackage{otf}
\usepackage{xspace}
\usepackage{newpxtext}
\usepackage[utf8]{inputenc} %中国語コンパイル環境-cjkホットショット
\usepackage{CJKutf8} %中国語コンパイル環境
\usepackage{okumacro} %漢字ruby
\renewcommand{\abstractname}{注意事項}
\newtagform{textbf}[	extbf]{[}{]}
\usetagform{textbf}
\newcommand*{\ie}{\textbf{\textit{i.e.}}\@\xspace}
\renewcommand{\qedsymbol}{$\blacksquare$}
\newtcbtheorem[]{reidai}{例題}
{fonttitle=\gtfamily\sffamily\bfseries\upshape\large,
colframe=black,colback=black!15!white,
rightrule=1pt,leftrule=1pt,bottomrule=2pt,
colbacktitle=black,theorem style=standard,breakable,arc=10pt}
{tha}
\renewcommand{\thefootnote}{\arabic{footnote}}
\newtheoremstyle{mystyle}%
  {}%                      % 上部スペース
  {}%                      % 下部スペース
  {}%                      % 本文フォント
  {}%                      % 1行目のインデント量
  {\bfseries}%             % 見出しフォント
  :%                       % 見出し後の句読点
  { }%                     % 見出し後のスペース
  {\thmname{#1}\thmnumber{ #2}\thmnote{ (#3)}}
\theoremstyle{mystyle}
% \setcounter{section}{0}
% \stepcounter{section}
% セクションカウンターを使用するが、表示はしない新しいセクションコマンドを作成
\newtheorem{dfn}{\texttt{Def.}}[section]
\newtheorem{exm}[dfn]{\texttt{Ex.}}
\newtheorem{prop}[dfn]{\texttt{Prop.}}
\newtheorem{lem}[dfn]{\texttt{Lem.}}
\newtheorem{thm}[dfn]{\texttt{Thm.}}
\newtheorem{cor}[dfn]{\texttt{Cor.}}
\newtheorem{rem}[dfn]{\texttt{Rem.}}
\newtheorem{fact}[dfn]{\texttt{Fact}}
\renewcommand{\qedsymbol}{$\blacksquare$}
\usepackage{lipsum} % 用于生成示例文本
\usepackage{float} % 强制浮动
\usepackage{tikz} % 用于定位
%排版
\newcommand{\kai}%解答
{\noindent
\begin{tikzpicture}[scale=0.2, baseline=2.8pt]
\draw (3.3,1) node{\large\textgt{解 答}};
\draw[thick, rounded corners=3pt,] (0,0)--(6.5,0)--(6.5,2.2)--(0,2.2)--cycle;
\end{tikzpicture}\;}
\newcommand{\shomei}%証明
{\noindent
\begin{tikzpicture}[scale=0.2, baseline=2.8pt]
\draw (3.3,1) node{\textgt{証 明}};
\draw[double,thick,rounded corners=3pt,] (0,0)--(6.5,0)--(6.5,2.2)--(0,2.2)--cycle;
\end{tikzpicture}\;}
%補足
\newcommand{\hosoku}{\noindent
\begin{tikzpicture}[scale=0.2, baseline=2.8pt]
\draw (6,1) node{\large\textgt{補足}};
\fill (0,1)--(1,0)--(2,1)--(1,2)--cycle;
\fill[gray] (1,1)--(2,0)--(3,1)--(2,2)--cycle;
\fill (2,1)--(3,0)--(4,1)--(3,2)--cycle;
\fill (10,1)--(11,0)--(12,1)--(11,2)--cycle;
\fill[gray] (9,1)--(10,0)--(11,1)--(10,2)--cycle;
\fill (8,1)--(9,0)--(10,1)--(9,2)--cycle;
\end{tikzpicture};}
\newcommand{\abb}[1]{%
\begin{tikzpicture}[baseline]
\node[draw=black, 
      rectangle, 
      minimum width=0.8cm, 
      minimum height=0.3cm, 
      fill=gray!25, 
      font=\bfseries,
      line width=1pt,
      inner sep=2pt,
      anchor=base] {#1};
\end{tikzpicture}%
}
\newcommand{\ab}[1]{%
\begin{tikzpicture}[baseline]
\node[draw=black, 
      rectangle, 
      minimum width=0.8cm, 
      minimum height=0.3cm, 
      font=\bfseries,
      line width=1pt,
      inner sep=2pt,
      anchor=base] {$#1$};
\end{tikzpicture}%
}

\newcommand{\maru}[1]{\tikz[baseline=-0.7ex]{
    \node[shape=circle,draw,inner sep=1pt,minimum size=5pt,anchor=center] {\footnotesize #1};}}
\definecolor{headercolor}{RGB}{220,220,220}
\definecolor{rowcolor1}{RGB}{245,245,245}
\definecolor{rowcolor2}{RGB}{255,255,255}
%注意
\newcommand{\chui}{\noindent
\begin{tikzpicture}[scale=0.2, baseline=2.8pt]
\fill (0,0)--(6.5,0)--(6.5,2.2)--(0,2.2);
\draw (3.3,1) node[white]{\large\textgt{注意!}};
\draw[thick] (0,0)--(6.5,0)--(6.5,2.2)--(0,2.2)--cycle;
\end{tikzpicture};}
\title{\vspace{-3cm} 蔡さんへのTouch連絡(25/07/24)}  %タイトル
\author{Linc\ -\ 伊}  %著者名
\date{}  %日付
\begin{document}
\maketitle
%\vspace{-0.4cm}
%\begin{figure}[H]
%\centering
%\begin{tikzpicture}[remember picture, overlay]
%   \node[anchor=north east] at (current page.north east) {%
%        \includegraphics[width=2cm]{pics/qr.png} % 修正图片地址
%    };
%    \node[anchor=north east, yshift=-2cm] at (current page.north east) {デジタル版はここ};
%\end{tikzpicture}
%\label{fig:my_label}
%\end{figure}
%\begin{abstract} %概要
  %注意事項
%\end{abstract}
%\begin{reidai}{2次方程式}{解答}
%\end{reidai}
%\begin{proof}
%\end{proof}
\section*{\textbf{問題}}
\noindent
複素数平面上の3点 $\mathrm{A}(\alpha)$, $\mathrm{B}(\beta)$, $\mathrm{C}(\gamma)$ を頂点とする三角形 $\mathrm{ABC}$ において

$$\frac{\gamma - \alpha}{\beta - \alpha} = 1 - i$$

であるとする。以下,偏角 $\theta$ の範囲は $0 \leq \theta < 2\pi$ とする。\\


\textbf{(1)}\quad 複素数 $\dfrac{\gamma - \alpha}{\beta - \alpha}$ を極形式で表すと

$$\dfrac{\gamma - \alpha}{\beta - \alpha} = \sqrt{\ab{\textsf N}} \left(\cos \dfrac{\ab{\textsf O}}{\ab{\textsf P}} \pi + i \sin \dfrac{\abb{\textsf O}}{\abb{\textsf P}} \pi\right)$$

である。よって,点 $\mathrm{C}$ は,点 $\mathrm{B}$ を点 $\mathrm{A}$ を中心として $\dfrac{\ab{\textsf Q}}{\ab{\textsf R}} \pi$ だけ回転し,さらに点 $\mathrm{A}$ からの距離を $\sqrt{\ab{\textsf S}}$ 倍した点である。これより,複素数 $w = \dfrac{\gamma - \beta}{\alpha - \beta}$ の絶対値と偏角は

$$|w| = \ab{\textsf T}, \quad \arg w = \dfrac{\ab{\textsf U}}{\ab{\textsf V}} \pi$$

\noindent
である。\\


\textbf{(2)}\quad $\alpha + \beta + \gamma = 0$ とすると

$$|\alpha| : |\beta| : |\gamma| = \sqrt{\ab{\textsf W}} : \sqrt{\ab{\textsf X}} : \sqrt{\ab{\textsf Y}}$$

である。
\newpage
\section*{\textbf{解答}}
\noindent
\textbf{(2)}\quad $\alpha + \beta + \gamma = 0$ より,$\gamma = -\alpha - \beta$ である。よって、\\

\noindent
$\dfrac{\gamma - \alpha}{\beta - \alpha} = 1-i\ \implies\ \dfrac{-2\alpha-\beta}{\beta - \alpha} = 1-i\ \implies\ -2\alpha - \beta = (1-i)(\beta - \alpha)$\\

\noindent
$\implies\ i\beta-2\beta = \alpha +i \alpha\ \implies\ \beta = \dfrac{(1+i)\alpha}{i-2}$\ となる。\\
\\ 

\noindent
よって、$|\alpha|:|\beta|=\left| \alpha \right|:\left|\dfrac{(1+i)}{i-2}\right|\cdot \left|\alpha\right|$ である。\\

$\because \left| i+1 \right| = \sqrt{1^2 + 1^2}=\sqrt{2}\ ,\  \left| i-2 \right| = \sqrt{(-2)^2 + 1^2}=\sqrt{5}$ \\

$\therefore \left| \alpha \right|:\left| \beta \right|=\left| \alpha \right|:\left|\dfrac{(1+i)}{i-2}\right|\cdot\left|\alpha\right|=\sqrt{5}:\sqrt{2}$\ となる。\ie $\left| \beta \right|=\dfrac{\sqrt{2}}{\sqrt{5}}\left| \alpha \right|$ \\

\noindent
よって、$\gamma = -\alpha - \beta=-\alpha-\dfrac{i+1}{i-2} \alpha =\dfrac{-2i+1}{i-2} \alpha$ である。\\
\\

\noindent
次に、$\left| \alpha \right|:\left| \gamma \right|=\left| \alpha \right|:\left|\dfrac{-2i+1}{i-2}\right|\cdot\left|\alpha\right|$ を求めればよい。\\

$\because \left| -2i+1 \right| = \sqrt{(-2)^2 + 1^2}=\sqrt{5}\ ,\  \left| i-2 \right| = \sqrt{(-2)^2 + 1^2}=\sqrt{5}$\\

$\therefore \left| \alpha \right|:\left| \gamma \right|=\left| \alpha \right|:\left|\dfrac{-2i+1}{i-2}\right|\cdot\left|\alpha\right|=1:1$\\
\\

\noindent
したがって、$|\alpha|:|\beta|:|\gamma|=|\alpha|:\dfrac{\sqrt{2}}{\sqrt{5}}\left| \alpha \right|:|\alpha|=\color{red}\sqrt{5}:\sqrt{2}:\sqrt{5}$ となる。\\

\end{document}