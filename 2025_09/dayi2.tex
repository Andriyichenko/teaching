\documentclass[dvipdfmx,a4paper,12pt]{jsarticle}
\usepackage{amsmath,amssymb,amsfonts,amsthm}
\usepackage[dvipdfmx]{graphicx}
\usepackage{tikz}
\usetikzlibrary{positioning, intersections, calc, arrows.meta, math, through}
\usepackage{tcolorbox}
\tcbuselibrary{theorems,breakable}
\usepackage[inline]{enumitem} % provides enumerate* (inline lists)
\usepackage{mathtools}
\usepackage{otf}
\usepackage{xspace}
\usepackage{newpxtext}
\usepackage[utf8]{inputenc} %中国語コンパイル環境-cjkホットショット
\usepackage{CJKutf8} %中国語コンパイル環境
\usepackage{okumacro} %漢字ruby
\renewcommand{\abstractname}{注意事項}
\newtagform{textbf}[	extbf]{[}{]}
\usetagform{textbf}
\newcommand*{\ie}{\textbf{\textit{i.e.}}\@\xspace}
\renewcommand{\qedsymbol}{$\blacksquare$}
\newtcbtheorem[]{reidai}{例題}
{fonttitle=\gtfamily\sffamily\bfseries\upshape\large,
colframe=black,colback=black!15!white,
rightrule=1pt,leftrule=1pt,bottomrule=2pt,
colbacktitle=black,theorem style=standard,breakable,arc=10pt}
{tha}
\renewcommand{\thefootnote}{\arabic{footnote}}
\newtheoremstyle{mystyle}%
  {}%                      % 上部スペース
  {}%                      % 下部スペース
  {}%                      % 本文フォント
  {}%                      % 1行目のインデント量
  {\bfseries}%             % 見出しフォント
  :%                       % 見出し後の句読点
  { }%                     % 見出し後のスペース
  {\thmname{#1}\thmnumber{ #2}\thmnote{ (#3)}}
\theoremstyle{mystyle}
% \setcounter{section}{0}
% \stepcounter{section}
% セクションカウンターを使用するが、表示はしない新しいセクションコマンドを作成
\newtheorem{dfn}{\texttt{Def.}}[section]
\newtheorem{exm}[dfn]{\texttt{Ex.}}
\newtheorem{prop}[dfn]{\texttt{Prop.}}
\newtheorem{lem}[dfn]{\texttt{Lem.}}
\newtheorem{thm}[dfn]{\texttt{Thm.}}
\newtheorem{cor}[dfn]{\texttt{Cor.}}
\newtheorem{rem}[dfn]{\texttt{Rem.}}
\newtheorem{fact}[dfn]{\texttt{Fact}}
\renewcommand{\qedsymbol}{$\blacksquare$}
\usepackage{lipsum} % 用于生成示例文本
\usepackage{float} % 强制浮动
\usepackage{tikz} % 用于定位
%排版
\newcommand{\kai}%解答
{\noindent
\begin{tikzpicture}[scale=0.2, baseline=2.8pt]
\draw (3.3,1) node{\large\textgt{解 答}};
\draw[thick, rounded corners=3pt,] (0,0)--(6.5,0)--(6.5,2.2)--(0,2.2)--cycle;
\end{tikzpicture};}
\newcommand{\shomei}%証明
{\noindent
\begin{tikzpicture}[scale=0.2, baseline=2.8pt]
\draw (3.3,1.2) node{\textgt{証 明}};
\draw[double,thick,rounded corners=3pt,] (0,0)--(6.5,0)--(6.5,2.4)--(0,2.4)--cycle;
\end{tikzpicture}}
%補足
\newcommand{\hosoku}{\noindent
\begin{tikzpicture}[scale=0.2, baseline=2.8pt]
\draw (6,1) node{\large\textgt{補足}};
\fill (0,1)--(1,0)--(2,1)--(1,2)--cycle;
\fill[gray] (1,1)--(2,0)--(3,1)--(2,2)--cycle;
\fill (2,1)--(3,0)--(4,1)--(3,2)--cycle;
\fill (10,1)--(11,0)--(12,1)--(11,2)--cycle;
\fill[gray] (9,1)--(10,0)--(11,1)--(10,2)--cycle;
\fill (8,1)--(9,0)--(10,1)--(9,2)--cycle;
\end{tikzpicture};}
\newcommand{\abb}[1]{%
\begin{tikzpicture}[baseline]
\node[draw=black, 
      rectangle, 
      minimum width=0.8cm, 
      minimum height=0.3cm, 
      fill=gray!25, 
      font=\bfseries,
      line width=1pt,
      inner sep=2pt,
      anchor=base] {#1};
\end{tikzpicture}%
}
\newcommand{\ab}[1]{%
\begin{tikzpicture}[baseline]
\node[draw=black, 
      rectangle, 
      minimum width=0.8cm, 
      minimum height=0.3cm, 
      font=\bfseries,
      line width=1pt,
      inner sep=2pt,
      anchor=base] {$#1$};
\end{tikzpicture}%
}

\newcommand{\maru}[1]{\tikz[baseline=-0.7ex]{
    \node[shape=circle,draw,inner sep=1pt,minimum size=5pt,anchor=center] {\footnotesize #1};}}
\definecolor{headercolor}{RGB}{220,220,220}
\definecolor{rowcolor1}{RGB}{245,245,245}
\definecolor{rowcolor2}{RGB}{255,255,255}
%注意
\newcommand{\chui}{\noindent
\begin{tikzpicture}[scale=0.2, baseline=2.8pt]
\fill (0,0)--(6.5,0)--(6.5,2.2)--(0,2.2);
\draw (3.3,1) node[white]{\large\textgt{注意!}};
\draw[thick] (0,0)--(6.5,0)--(6.5,2.2)--(0,2.2)--cycle;
\end{tikzpicture};}
\title{\vspace{-3cm} 陳旭銘さんへの質疑応答}  %タイトル
\author{Linc\ -\ 伊}  %著者名
\date{}  %日付
\begin{document}
\maketitle
%\vspace{-0.4cm}
%\begin{figure}[H]
%\centering
%\begin{tikzpicture}[remember picture, overlay]
%   \node[anchor=north east] at (current page.north east) {%
%        \includegraphics[width=2cm]{pics/qr.png} % 修正图片地址
%    };
%    \node[anchor=north east, yshift=-2cm] at (current page.north east) {デジタル版はここ};
%\end{tikzpicture}
%\label{fig:my_label}
%\end{figure}
%\begin{abstract} %概要
  %注意事項
%\end{abstract}
%\begin{reidai}{2次方程式}{解答}
%\end{reidai}
%\begin{proof}
%\end{proof}
\section*{\textbf{\text{III}の解答}}
\noindent
\textbf{(1)}\quad $a,b$ を素因数分解する:
\[
a=588=2^2\cdot3\cdot7^2,\qquad
b=1260=2^2\cdot3^2\cdot5\cdot7.
\]
よって, $\gcd(a,b)=2^{\color{red}{2}}\cdot3\cdot {\color{red}{7}}^{\color{red}{2}}={\color{red}{84}}$となる。 \quad ここで、 $ \gcd(a,b) $は $a,b$ の公約数のうち最大のものを表す。また, 
$\mathrm{lcm}(a,b)=2^2\cdot3^2\cdot5\cdot7^2=1260$となる。 \quad ここで、 $ \mathrm{lcm}(a,b) $は $a,b$ の公倍数のうち最小のものを表す. \\


\noindent
\textbf{(2)}\quad
下の2つの条件を満たす正の整数$c$ を考える。
\begin{enumerate}[label=(\roman*)]
  \item $\gcd(a,b,c)=\gcd(a,b)$.
  \item $\mathrm{lcm}(a,b,c)=4 \mathrm{lcm}(a,b)$.
\end{enumerate}

\noindent
そこで $c$ を(ii)の条件より、$c$の素因数$2$の個数は$4$個であることがわかる。また、(i)と(ii)の条件より、$c$の素因数$3$の個数は$1$個\ or\ $2$個であり、$c$の素因数$5$の個数は$0$個\ or\ $1$個であり、$c$の素因数$7$の個数は$1$個\ or\ $2$個である。したがって、積の法則により、$c$は全部で$2\times2\times2={\color{red}{8}}$通り存在する。そのような$c$の中で最小のものは$2^4\cdot3^1\cdot5^0\cdot7^1={\color{red}{336}}$である。\\[1em]

\textbf{(3)}\quad
$ax-by=336$に対し,$a=588,\ b=1260$ を代入して
\[
588x-1260y=336.\quad \cdots (1)
\]
両辺を $\gcd(588,1260)=84$ で割ると7x-15y=4となる。ここで、$x$,$y$は整数解である。よって、${\color{red}{7}} x-{\color{red}{15}} y={\color{red}{4}}\ \implies\ 7x=15y+4\implies$\\[0.5em]
$\begin{cases}y=1,\ 7x=19(条件満たさない)\\y=2,\ 7x=34(条件満たさない)\\y=3,\ 7x=49\ \implies\ x=7\end{cases}$\\[0.5em]
となる。よって、$x={\color{red}{7}},\ y={\color{red}{3}} $は$x,y$の整数解のうち、$y$が最小の正整数解である。\\[1em]

\noindent
よって、$\begin{cases}7x-15y=4\\7\cdot 7-15\cdot 3=4\end{cases}$\ より、$7(x-7)-15(y-3)=0\ \implies\ 7(x-7)=15(y-3)$、$7$と$15$は互いに素であるから$\begin{cases}x-7=15k\\y-3=7k\end{cases}$ $\implies$ $\begin{cases}x={\color{red}{7+15k}}\\y={\color{red}{3+7k}}\end{cases}$、ただし、$k$は整数。
\end{document}