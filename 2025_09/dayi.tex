\documentclass[dvipdfmx,a4paper,12pt]{jarticle}
\usepackage{amsmath,amssymb,amsfonts,amsthm}
\usepackage[dvipdfmx]{graphicx}
\usepackage{tikz}
\usetikzlibrary{positioning, intersections, calc, arrows.meta, math, through}
\usepackage{tcolorbox}
\tcbuselibrary{theorems,breakable}
\usepackage[inline]{enumitem} % provides enumerate* (inline lists)
\usepackage{mathtools}
\usepackage{otf}
\usepackage{xspace}
\usepackage{newpxtext}
\usepackage[utf8]{inputenc} %中国語コンパイル環境-cjkホットショット
\usepackage{CJKutf8} %中国語コンパイル環境
\usepackage{okumacro} %漢字ruby
\renewcommand{\abstractname}{注意事項}
\newtagform{textbf}[	extbf]{[}{]}
\usetagform{textbf}
\newcommand*{\ie}{\textbf{\textit{i.e.}}\@\xspace}
\renewcommand{\qedsymbol}{$\blacksquare$}
\newtcbtheorem[]{reidai}{例題}
{fonttitle=\gtfamily\sffamily\bfseries\upshape\large,
colframe=black,colback=black!15!white,
rightrule=1pt,leftrule=1pt,bottomrule=2pt,
colbacktitle=black,theorem style=standard,breakable,arc=10pt}
{tha}
\renewcommand{\thefootnote}{\arabic{footnote}}
\newtheoremstyle{mystyle}%
  {}%                      % 上部スペース
  {}%                      % 下部スペース
  {}%                      % 本文フォント
  {}%                      % 1行目のインデント量
  {\bfseries}%             % 見出しフォント
  :%                       % 見出し後の句読点
  { }%                     % 見出し後のスペース
  {\thmname{#1}\thmnumber{ #2}\thmnote{ (#3)}}
\theoremstyle{mystyle}
% \setcounter{section}{0}
% \stepcounter{section}
% セクションカウンターを使用するが、表示はしない新しいセクションコマンドを作成
\newtheorem{dfn}{\texttt{Def.}}[section]
\newtheorem{exm}[dfn]{\texttt{Ex.}}
\newtheorem{prop}[dfn]{\texttt{Prop.}}
\newtheorem{lem}[dfn]{\texttt{Lem.}}
\newtheorem{thm}[dfn]{\texttt{Thm.}}
\newtheorem{cor}[dfn]{\texttt{Cor.}}
\newtheorem{rem}[dfn]{\texttt{Rem.}}
\newtheorem{fact}[dfn]{\texttt{Fact}}
\renewcommand{\qedsymbol}{$\blacksquare$}
\usepackage{lipsum} % 用于生成示例文本
\usepackage{float} % 强制浮动
\usepackage{tikz} % 用于定位
%排版
\newcommand{\kai}%解答
{\noindent
\begin{tikzpicture}[scale=0.2, baseline=2.8pt]
\draw (3.3,1) node{\large\textgt{解 答}};
\draw[thick, rounded corners=3pt,] (0,0)--(6.5,0)--(6.5,2.2)--(0,2.2)--cycle;
\end{tikzpicture};}
\newcommand{\shomei}%証明
{\noindent
\begin{tikzpicture}[scale=0.2, baseline=2.8pt]
\draw (3.3,1.2) node{\textgt{証 明}};
\draw[double,thick,rounded corners=3pt,] (0,0)--(6.5,0)--(6.5,2.4)--(0,2.4)--cycle;
\end{tikzpicture}}
%補足
\newcommand{\hosoku}{\noindent
\begin{tikzpicture}[scale=0.2, baseline=2.8pt]
\draw (6,1) node{\large\textgt{補足}};
\fill (0,1)--(1,0)--(2,1)--(1,2)--cycle;
\fill[gray] (1,1)--(2,0)--(3,1)--(2,2)--cycle;
\fill (2,1)--(3,0)--(4,1)--(3,2)--cycle;
\fill (10,1)--(11,0)--(12,1)--(11,2)--cycle;
\fill[gray] (9,1)--(10,0)--(11,1)--(10,2)--cycle;
\fill (8,1)--(9,0)--(10,1)--(9,2)--cycle;
\end{tikzpicture};}
\newcommand{\abb}[1]{%
\begin{tikzpicture}[baseline]
\node[draw=black, 
      rectangle, 
      minimum width=0.8cm, 
      minimum height=0.3cm, 
      fill=gray!25, 
      font=\bfseries,
      line width=1pt,
      inner sep=2pt,
      anchor=base] {#1};
\end{tikzpicture}%
}
\newcommand{\ab}[1]{%
\begin{tikzpicture}[baseline]
\node[draw=black, 
      rectangle, 
      minimum width=0.8cm, 
      minimum height=0.3cm, 
      font=\bfseries,
      line width=1pt,
      inner sep=2pt,
      anchor=base] {$#1$};
\end{tikzpicture}%
}

\newcommand{\maru}[1]{\tikz[baseline=-0.7ex]{
    \node[shape=circle,draw,inner sep=1pt,minimum size=5pt,anchor=center] {\footnotesize #1};}}
\definecolor{headercolor}{RGB}{220,220,220}
\definecolor{rowcolor1}{RGB}{245,245,245}
\definecolor{rowcolor2}{RGB}{255,255,255}
%注意
\newcommand{\chui}{\noindent
\begin{tikzpicture}[scale=0.2, baseline=2.8pt]
\fill (0,0)--(6.5,0)--(6.5,2.2)--(0,2.2);
\draw (3.3,1) node[white]{\large\textgt{注意!}};
\draw[thick] (0,0)--(6.5,0)--(6.5,2.2)--(0,2.2)--cycle;
\end{tikzpicture};}
\title{\vspace{-3cm} 陳旭銘さんへの質疑応答}  %タイトル
\author{Linc\ -\ 伊}  %著者名
\date{}  %日付
\begin{document}
\maketitle
%\vspace{-0.4cm}
%\begin{figure}[H]
%\centering
%\begin{tikzpicture}[remember picture, overlay]
%   \node[anchor=north east] at (current page.north east) {%
%        \includegraphics[width=2cm]{pics/qr.png} % 修正图片地址
%    };
%    \node[anchor=north east, yshift=-2cm] at (current page.north east) {デジタル版はここ};
%\end{tikzpicture}
%\label{fig:my_label}
%\end{figure}
%\begin{abstract} %概要
  %注意事項
%\end{abstract}
%\begin{reidai}{2次方程式}{解答}
%\end{reidai}
%\begin{proof}
%\end{proof}
\section*{\textbf{問題\, 1}}
\vspace{0.5cm}
\begin{enumerate}
\item $(-x+2y+3z)(2x-3y+4z)(3x+4y-5z)$ を展開したとき、$xyz$ の係数は \ab{\textsf{ABC}} である。\\[0.5em]

\item $\displaystyle \frac{479}{700}$ を小数で表したとき、小数第 $2023$ 位の数字は \ab{\textsf{D}} である。\\[0.5em]

\item 循環小数を次のように書き表すことにする。

例 $0.121212\cdots=0.\dot{1}\dot{2}$  $0.345345345\cdots=0.\dot{3}\dot{4}\dot{5}$

$a = 0.228,\ \ b=0.335244$ のとき,$x=2a-b$ を循環小数で表すと
\[
x=0.\dot{\ab{\textsf{E}}}\dot{\ab{\textsf{F}}}
\]
である。ここで

\[
100x = \ab{\textsf{GH}} + x
\]

であることに注意すると、$x$ は既約分数として

\[
x = \frac{\ab{\textsf{I}}}{\ab{\textsf{JK}}}
\]

と表される。
\end{enumerate}

\newpage
\section*{\textbf{解答}}
(1) \quad $(-x)(-3y)(-5z) + (-x)(4z)(4y) + (2y)(2x)(-5z) + (2y)(4z)(3x) \\
\quad + (3z)(2x)(4y) + (3z)(-3y)(3x) = {\color{red}{-30}}xyz$\\[0.5em]

(2)\quad $\dfrac{479}{700} = 0.68\dot{4}2857\dot{1} $であって、小数第3位から循環であり、$(2023-2) \div 6 = 336 \cdots 5 $である。すなわち、小数第2023位は5番目の数字であるから、答えは$\color{red}7$である。\\[0.5em]

(3) \quad $a = 0.228$ より、 $2a = 0.456$である。また、$b = 0.\dot{3}3524\dot{4}$ であるから、$2a-b$を循環小数で表すと、$x=2a-b=0.\color{red}\dot{1}\dot{2}$である。ここで、$100x=12.121212\cdots$であるので、$100x={\color{red}{12}}+x$
より、$99x=12$より、$x=\dfrac{12}{99}=\color{red}\dfrac{4}{33}$である。

\newpage
\section*{\textbf{問題\, 2}}
$x$ についての2次不等式

\begin{align*}
&x^2 - 3x + 2 < 0 \quad \cdots\cdots \quad \maru{1} \\
&x^2 - 2ax - 3a^2 < 0 \quad \cdots\cdots \quad \maru{2}
\end{align*}

を考える。

\begin{enumerate}
\item 不等式 \maru{1} の解は \ab{\textsf{L}} $< x <$ \ab{\textsf{M}} である。

\item 次の文中の \ab{\textsf{O}} $\sim$ \ab{\textsf{Q}} には、下の選択肢 \maru{0} $\sim$ \maru{9} の中から適するものを選びなさい。また、\ab{\textsf{N}} には、適する数を入れなさい。

不等式 \maru{2} の解は

\begin{align*}
a > \ab{\textsf{N}} \text{のとき} \quad &\ab{\textsf{O}} \\
a = \abb{\textsf{N}} \text{のとき} \quad &\ab{\textsf{P}} \\
a < \abb{\textsf{N}} \text{のとき} \quad &\ab{\textsf{Q}}
\end{align*}

である。


\begin{tabular}{l@{\hspace{2cm}}l}
\maru{0} \quad $a < x < -3a$ & \maru{5} \quad $-3 < x < a^2$ \\[0.2em]
\maru{1} \quad $-a < x < 3a$ & \maru{6} \quad $-3a < x < -a$ \\[0.2em]
\maru{2} \quad $-2a < x < 3a$ & \maru{7} \quad $3a < x < a$ \\[0.2em]
\maru{3} \quad $1 < x < 3a^2$ & \maru{8} \quad すべての実数 \\[0.2em]
\maru{4} \quad $3a < x < -a$ & \maru{9} \quad 解はなし
\end{tabular}
\end{enumerate}
\newpage
\section*{\textbf{解答}}
(1) \quad $x^2 - 3x + 2 < 0 $より、$(x-1)(x-2) < 0$となる。したがって、不等式\maru{1}の解は $ 1 < x < 2 $である。\\[0.5em]

(2) \quad $x^2 - 2ax - 3a^2 < 0 \quad \text{より} \quad (x-3a)(x+a) < 0$となるが、$a$の値によって場合分けを考える必要がある。
まず、$(x-3a)(x+a) = 0$を解くと、その2つの解は$3a,-a$である。したがって、不等式\maru{2}の解は
\begin{align*}
\begin{cases}
a > {\color{red}0} \text{のとき}    & \color{red}-a < x < 3a \,\maru{1}\\
a = {\color{red}0} \text{のとき}   & x^2 < 0\quad\text{\color{red}解はなし\,\maru{9}}\\
a < {\color{red}0} \text{のとき}   & \color{red}3a < x < -a \,\maru{4}
\end{cases}
\end{align*}


\end{document}